\section{Conclusion}

This paper presented Helios, a path-driven execution engine that accelerates EVM transaction processing while preserving gas-semantic correctness. On modern interpreters such as Revm, three barriers impede meaningful acceleration. The \textit{Compilation Limitation} arises when aggressive optimizations violate gas accounting. The \textit{Optimization Dilemma} emerges when instrumentation costs exceed execution latency. The \textit{Granularity Mismatch} occurs when transaction-level optimization yields ephemeral artifacts. Helios addresses these challenges through \textit{hybrid execution} that restricts optimization to static-cost instructions, \textit{asynchronous tracing} that decouples profiling from the critical path, and \textit{frame-level caching} that exploits call locality while resisting cache-flooding attacks.

Helios advances the state of the art along four dimensions. It ensures \textit{safety} by maintaining gas-semantic equivalence with canonical EVM, improves \textit{efficiency} by removing instrumentation from the critical path, provides \textit{robustness} by resisting adversarial path explosion, and demonstrates \textit{versatility} by serving both validators and archive nodes from a unified artifact set.

Evaluation on Ethereum mainnet confirms the effectiveness of this design. Helios achieves 6.60$\times$ median speedup in Replay Mode and 2.05$\times$ in Online Mode with 4.9\% storage overhead. Future work will explore selective JIT compilation, frame-level fallback mechanisms, and hybrid cache granularity. By decoupling optimization from the critical path, Helios offers a foundation for scalable EVM infrastructure.
