\section{Conclusion}

We presented Helios, a path-driven execution engine designed to accelerate transaction processing on high-performance EVM clients. We identified an optimization dilemma in existing acceleration strategies. On modern, highly optimized interpreters, the overhead of detailed tracing and artifact management often negates the benefits of optimization. Helios addresses this challenge through a novel architecture that combines lightweight asynchronous tracing with frame-level caching. By restricting optimization to static-cost instructions, Helios achieves gas-semantic equivalence by construction and eliminates the economic risks associated with aggressive JIT compilation.

Our evaluation on Ethereum mainnet workloads demonstrates the efficacy of this approach. In Replay Mode, Helios achieves a median speedup of 6.60$\times$ over the baseline Revm interpreter, validating its potential for accelerating archive node synchronization and historical data analysis. In Online Mode, Helios effectively accelerates hot execution paths, leveraging the strong path locality inherent in smart contract execution.

Helios establishes a new design point for blockchain execution engines, prioritizing safety, correctness, and architectural simplicity alongside raw performance. Future work will explore integrating JIT compilation to further reduce interpretation overhead for optimized paths and refining the speculative execution model with frame-level fallback mechanisms to maximize coverage. By decoupling optimization from the critical path, Helios provides a scalable foundation for the next generation of EVM infrastructure.