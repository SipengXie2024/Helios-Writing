\begin{abstract}
Smart contract execution stands as a critical scalability bottleneck for EVM-based blockchains. Prevalent acceleration strategies, notably JIT compilation and path-driven speculative execution, face fundamental limitations. JIT approaches compromise gas accounting determinism and introduce security vulnerabilities exploitable by pathological code patterns. Conversely, path-driven methods encounter an "optimization dilemma" on hyper-optimized clients like Revm, where the overhead of tracing and artifact management negates execution gains. Furthermore, their reliance on transaction-level granularity results in ephemeral artifacts with limited reuse due to combinatorial path explosion.

We present Helios, a path-driven execution engine that overcomes these barriers through three architectural principles. First, hybrid execution restricts optimization to static-cost instructions, guaranteeing gas-semantic equivalence by construction. Second, asynchronous tracing decouples profiling from the critical path, thereby eliminating instrumentation bottlenecks. Third, frame-level caching exploits the strong path locality of individual contract calls, transforming ephemeral traces into reusable artifacts. The resulting SSA graphs are executed by a guarded register-based interpreter that ensures correctness via native fallback upon path divergence.

Evaluation on the Ethereum mainnet demonstrates a median speedup of 6.60$\times$ in Replay Mode. In Online Mode, frequency-based filtering achieves 2.05$\times$ acceleration with a negligible 4.9\% storage overhead. These results establish Helios as a scalable and safe foundation for next-generation EVM infrastructure.
\end{abstract}